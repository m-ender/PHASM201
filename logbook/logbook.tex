\documentclass[a4paper,onecolumn,11pt]{report}

% Use for onecolumn layout
\usepackage[left=3cm,right=3cm,top=3cm,bottom=3cm]{geometry}

\usepackage{graphicx}

\usepackage[utf8]{inputenc}
\usepackage[hyphens]{url}

\usepackage{natbib}
\usepackage{textcomp}
\usepackage[greek,english]{babel}
\usepackage{amsmath}

\frenchspacing
%\renewcommand{\thefootnote}{\alph{footnote}}

\renewcommand{\vec}[1]{\ensuremath{\mathbf{#1}}}
\newcommand{\uvec}[1]{\ensuremath{\vec{\hat{#1}}}}
% Different command for lower-case greek letters as they need a different command for bold letters.
\newcommand{\gvec}[1]{\ensuremath{\boldsymbol{#1}}}
\newcommand{\guvec}[1]{\ensuremath{\gvec{\hat{#1}}}}
\newcommand{\dif}{\mathrm{d}}

\begingroup
\lccode`\~=`\ %
\lowercase{%
  \gdef\assignment{\setcounter{word}{0}%
    \catcode`~=\active
    \def~{\space\stepcounter{word}}}}%
\endgroup
\newcounter{word}
\def\endassignment{\stepcounter{word}%
  \begin{flushright}%
  (\arabic{word} words)%
  \end{flushright}%
}

\title{Accurate methods for computing rotation-dominated flows\\\small Logbook}

\author{Martin Büttner, University College London}
\date{\today}

\begin{document}

\maketitle

\subsubsection*{27 October 2014}

After finishing the literature survey and project outline, I set up the basic infrastructure for the project. All the data, code and written work will be maintained in a private \texttt{git}\footnote{\url{http://git-scm.com/}} repository at \url{https://github.com/mbuettner/PHASM201}. I've also set up individual \LaTeX\ projects for the literature survey, final report and this logbook.

The coming month will be spent on reading papers about existing well-balanced methods in-depth to evaluate which ones are worth implementing during the second phase of the project.

\subsubsection*{29 October 2014}

Read \citet{leveque1998balancing} today. He balances the SWEs with source terms by introducing another Riemann problem in the centre of the cell, such that each cell doesn't just have an its average value $Q_i$ but two values $Q_i \pm \delta$, such that the overall cell averages is the same. The Riemann problems at the cell boundaries are then solved between these new values. The trick is that the step inside the cell is chosen such that the resulting flow cancels exactly with the source term - hence this Riemann problem does not even need to be solved. Works up to Courant number 1. Caveat of the paper: the ``exact'' reference solutions used to compare their method with a fractional step method also use their method, just on a much finer grid. The paper explicitly mentions that this method \emph{only} captures quasi-steady solutions and does not perform well for solutions away from equilibrium. In particular, it is not suited to setting up transcritical flow problems. The paper suggests using other (possibly split) methods to find an approximation to a (quasi-)steady solution and then switch methods.

This method simplifies boundary conditions in the presence of source terms. The approach is also adapted to two dimensions in the paper.

\subsubsection*{4 November 2014}

I found another early paper, \citet{ambrosi1995approximation}, which observes the problems of unbalanced schemes, namely that still-lake equilibria are not preserved.

\citet{rogers2001adaptive} use a different formulation of the SWEs to balance their solver. Instead of writing the SWEs in terms of $(h, uh, vh)$, they use $(\zeta, uh, vh)$, where $\zeta$ is the disturbance of the water depth from still water level. The advantage is that still-lake equilibria don't contribute any flux terms in this case (while they do in the conventional formulation, due to differing water depth). Note that this method is tailored specifically to source terms due to variable bathymetry. They follow this up in \citet{rogers2003mathematical}. This paper generalises the method to arbitrary source terms and equilibria, but the method does seem to require information about the desired equilibrium to be included in the formulation of the SWEs themselves (meaning that no one solver can be used for several equilibria or will be particularly well suited to systems away from equilibrium). Generally speaking, the core of the method is to rewrite the entire system in variables which are deviations from one particular equilibrium.

\subsubsection*{7 November 2014}

Reading \citet{chertockwell}. They focus on the particular case, we're interested in: well-balanced schemes for the SWEs with variable bathymetry and Coriolis force. They are referencing quite a lot of papers with the same focus, many of which I haven't encountered during the literature survey. It might be worth having a look into those, too. Notably, this paper doesn't use Riemann--problem based methods, but a so-called ``central upwind'' method and an ``evolution Galerkin'' method. I'll look into these, and figure out whether they can be implemented with Clawpack as well.

The basis of their scheme is a piecewise \emph{linear} reconstruction, which might indicate that this is not suitable for Godunov-type methods. It might be possible to implement it with Clawpack anyway.

The schemes use Courant numbers 0.5 and 0.6, but the paper makes no mention of what the actual CFL condition is. This might be detailed in the referenced papers.

Another useful takeaway from the paper might be the numerical simulations used for verification, as their area of interest is the same as ours.

\subsubsection*{19 November 2014}

I've had a look at \citet{walters2009useful}. It seems to be dealing with completely different numerical schemes for the SWEs, like implicit and explicit Euler schemes. It does focus on the Coriolis force and seems to include bathymetry, but I think this might be beyond the scope of the project, as these are not FVMs.

\citet{audusse2009conservative} does not present a well-balanced scheme but derives a different numerical scheme that ensures conservativity in rotating frames, by applying the change of variables into the rotating frame \emph{after} discretisation. However, they close their paper saying that they wanted to extend their work to a well-balanced scheme ``in the near future''. I believe that follow-up paper is \citet{audusse2011preservation}, which I still have to look into.

\subsubsection*{22 November 2014}

Looking through other recent papers, they all focus on well-balanced methods that support wet/dry fronts. Hence, I'll be ignoring those for now and look into somewhat older papers instead.

\citet{garcia2000numerical} focuses primarily on geometrical source terms, which is also not particularly relevant to this project.

\subsubsection*{24 November 2014}

\citet{hubbard2000flux} present a general way to balance source terms, by including them in the numerical scheme used for the flux terms. They report that their method works for equilibria \emph{and} away from equilibria, but in the latter case slope limiters put a very strict bound on the Courant number (they used 0.1 in the paper). The paper itself seems very technical, so I'll get back to it after reading some more about the basics of the Roe solver.

Hence, I decided to start a detailed reading of \citet{toro2001shock} for next few days.

\subsubsection*{30 December 2014}

I ended up re-reading \citet{leveque2002finite} instead, which did greatly improve my understanding of the methods in general (including the Roe solver). An important decision I made with Prof. Johnson in the meantime is to restrict the methods to the dimensionally-split SWEs (in one dimension) for now. Since we were going to use a dimensional split for the 2D equations anyway, adding the second dimension does not actually provide anything new for studying the methods. Solving the equations in only one dimension instead has the advantage of running much faster (which allows for higher resolutions to be used), and makes it much easier to find interesting equilibrium states to test the methods. Furthermore, there is no necessity for AMR when restricting the tests to one dimension.

I also found out that GeoClaw uses an f-wave based solver, which is well-balanced with respect to the bathymetry terms. It is not entirely clear whether it is also well-balanced with respect to the Coriolis terms. See \citet{berger2011geoclaw}. I might look into this, although the capabilities of the particular solver used by GeoClaw go well beyond the scope of this project (the solver is positivity-preserving and includes bed friction) -- it also has the major disadvantage of working with dimensional quantities, which makes it harder to compare to the non-dimensionalised solvers I intend to write myself.

\subsubsection*{2 January 2015}

I've spent some time setting up Clawpack on a local VM and made sure it works properly.

\subsubsection*{4 January 2015}

I've modified the 1D Roe solver provided with Clawpack to solve the $x$-split 2D SWEs instead. I've got handwritten workings of this, which I could use for the write-up of the final report. I also added the Coriolis and bathymetry source terms via a source-splitting --- they are computed by a second-order Runge--Kutta method. In addition, I've prepared some plotting routines which take the bathymetry into account.

Computing the bathymetry source term via source-splitting is, of course, not well-balanced, but should act as the naive implementation for comparison. In fact, a simple Gaussian hump under a still lake is enough to produce unphysical oscillations within a very short time.

\subsubsection*{6 January 2015}

I had an extended meeting with Prof. Johnson today, in which we discussed the parameterisation of the equations, as well as relevant setups for numerical validation. We settled on the following:

\begin{itemize}
  \item Still lake with bathymetry and rotation.
  \item Steady rotating flow over a ridge, looking at subcritical, supercritical and transcritical flows. See \citet{esler2005steady}.
  \item Non-trivial geostrophic equilibrium with and without bathymetry.
  \item Wave over a ridge, as a perturbation about equilibrium.
  \item (Rossby adjustment problem.)
\end{itemize}

If we extend the validations to two dimensions later on, possible models are:

\begin{itemize}
  \item Uniform subcritical flow over a hump (under rotation). Exact solutions can be found with an elliptical solver.
  \item Axisymmetric problems, which can be solved in one (radial) dimension, but which the hyperbolic solvers would treat as generic two-dimensional problems.
  \item For more ideas, see \citet{rogers2001adaptive}.
\end{itemize}

\subsubsection*{10 January 2015}

The last few days were spent on implementing the above test models, as well as improving the framework to be more flexibly configurable. The main decision I made was to allow for the bathymetry and the initial condition on the flow variables to be specified independently. This gives more flexibility in testing only individual balances. I also fixed a bug in the Runge--Kutta solver, which made the unbalanced oscillations much larger than they were supposed to be. Oscillations still appear on a still lake, but they are much smaller, and the grid resolution can be decreased considerably before they are visible on the scale of the typical water depth.

The only initial conditions I haven't implemented yet are those found in \citet{esler2005steady}, which I will focus on next week.

\subsubsection*{27 January 2015}

I spent last week preparing for the progress interview and implementing the final initial conditions. In that process I hit a minor roadblock, when the supposedly stable flows wouldn't settle and instead carry away most of the momentum out of the domain. Prof. Johnson suggested to introduce a transversal pressure gradient, such that the Coriolis force originating from the initial velocity is balanced away from the obstacle. This is equivalent to towing the obstacle through water in solid-body rotation and then switching into the rest-frame of the obstacle.

This did indeed solve the problem, such that now the domain settles into a steady state within a very short time frame for most initial conditions. I decided to implement this a single parameterisable initial condition, instead of three for subcritical, transcritical and supercritical flow. This is a) because the velocities for these regimes depend on the Coriolis parameter, b) the phase space is a lot more complicated than these three cases --- in particular there are several transcritical states, as shown in \citet{esler2005steady}. By allowing to configure the background velocity manually, any point in phase space can be chosen to be examined.

With this in place, I was able to set up subcritical and supercritical flows both with and without rotation, as several qualitatively different transcritical solutions. I will now start to implement the first of the balanced-solvers.

\subsubsection*{16 February 2015}

Over the last few weeks, I have adapted the solver presented in \citet{leveque1998balancing} to include rotation (I've got handwritten workings on this, which I'll use for the final report), implemented it in Fortran and ran some tests on the model problems. The solver does work better than the unbalanced solver, but not as well as I expected. I'll have to look into whether there's a mistake in the code, or whether the results are actually correct.

As LeVeque noted, the Newton solver fails on transcritical problems. Prof. Johnson suggested to try solving the cubic equation in question exactly. This implementation performs identically for subcritical and supercritical problems, but does indeed not break down for transcritical problems. However, the results in cases where the Newton solver fails don't seem reliable as they introduce a lot more (stationary) shocks than the unbalanced solver on a fine grid. For now, I'll write this off as a shortcoming of the method, but I might revisit this once I've implemented the other methods.

Next I will try adapting the method presented in \citet{rogers2003mathematical} to our problems and implementing this solver next.

\addcontentsline{toc}{chapter}{Bibliography}
\bibliographystyle{plainnat}
\bibliography{../common/bibliography}

\end{document}
