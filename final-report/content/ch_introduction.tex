%!TEX root = ../final-report.tex
\chapter{Introduction}
\label{ch:introduction}

The shallow water equations (SWEs) are a simplified but very effective model for incompressible fluid flow. Despite the name, this model has been particularly successful in modelling large-scale geophysical flows of atmospheric and oceanic currents --- here, ``shallow'' merely means that fluid depth is small compared to the horizontal length scales of the system. Considering flows on the scale of the Earth, which can easily extend across several thousand kilometres, the roughly 4 kilometre deep ocean is indeed comparably shallow.

In the context of these applications, two effects are particularly important to model: the Coriolis force, due to the rotation of the Earth, as well as varying topography\footnote{Undersea topography is usually referred to as \emph{bathymetry}. This term will be used predominantly throughout this report.}. This report is concerned with simulating such models using \emph{finite volume methods}. These will usually include the above effects as source terms. Computational difficulties arise in steady flow when these very large forces balance pressure gradients exactly. This is referred to as a geostrophic balance, and large-scale real-world currents are close to this balance at all times. Therefore, the interest of this report lies in numerical methods which work around these computational problems such that they preserve geostrophic flows exactly and can reliably compute small perturbations of these steady states.

Two existing methods to account for such a balance in the presence of bathymetry are reviewed and extended to include the Coriolis terms as well. Both new methods are then evaluated on various systems to allow a comparison of their individual merits.

\section{Literature Overview}

The SWEs are a hyperbolic system of conservation laws. A lot of effort has been put into studying and solving these systems of equations numerically. Several textbooks about the numerical methods considered in this report exist, including \citet{leveque1992numerical}, \citet{toro1999riemann} and \citet{leveque2002finite}, as well as \citet{toro2001shock} which focuses on the SWEs in particular. In addition, there is a textbook on the rotating SWEs by \citet{zeitlin2007nonlinear}, Chapter 4 of which focuses on nu merical methods.

Such hyperbolic systems can be solved using finite volume methods in which the domain is divided up into a (not necessarily regular) grid of control volumes, and the conserved quantities are discretised by assuming that they are constant across each such volume. Note that ``volume'' is being used in a generalised sense here --- for a two-dimensional system like the SWEs, the grid cells are actually areas. The most popular of these methods is due to \citet{godunov1959difference}\footnote{The author could not obtain an English translation of this Russian paper, but the method developed by Godunov has been extensively reiterated in many papers and textbooks. Therefore, the explanation of the method in this report is based on those secondary sources.} and is simply known as Godunov's method --- in fact, today a whole family of Godunov methods has been developed based on the concepts derived in this original paper.

A fairly recent review of Godunov-type methods was conducted by \citet{toro2007godunov} and an older review can be found in \citet{sweby2001godunov}.

The piecewise-constant discretisation reduces the problem to a number of step functions at the cell boundaries. The initial-value problem consisting of a single step between two constant values is known as a Riemann problem (see \citet{toro1999riemann}, Section 2.2.2, \citet{leveque2002finite}, Section 3.8, or \citet{toro2007godunov}, Section 2.1) and can be solved exactly for many systems. The discretised SWEs can be solved by solving each of the Riemann problems at the cell boundaries. The other insight of Godunov's method is that each hyperbolic system has a number of characteristic waves which only propagate in certain directions at finite speeds, which allows to simplify the computation by only looking at waves that can propagate \emph{into} each cell. This technique is known as \emph{upwinding} (see e.g. \citet{leveque2002finite}, p. 72).

While it is possible to solve the Riemann problem exactly for many systems (including the SWEs), this dominates the computations required to solve each time step. Therefore, approximate solvers have been developed, the most popular one being due to \citet{roe1981approximate}.

However, these methods generally have been developed for homogeneous systems. A simple approach to account for source terms like bathymetry and the Coriolis force is to compute these independently of the homogeneous system in a separate step. This leads to problems in steady or quasi-steady scenarios, where the flux terms arising from the homogeneous system and the source terms are balanced at all times. According to \citet{toro2007godunov}, the first authors to recognise this were \citet{glimm1984generalized}. Preserving such a balance requires that a time step in the homogeneous system is  cancelled exactly by the corresponding time step for the source terms. Since these terms can in principle be very large, even for balanced systems, due to different methods being employed and cancelling terms being computed in separate steps, this is practically impossible. Hence, equilibria cannot be modelled accurately, even when the system is as simple as a still lake. The numerical errors create spurious oscillations which may even be amplified in future time steps. Furthermore, small perturbations away from equilibrium would be completely dominated by said numerical errors. These problems are particularly relevant for large-scale geophysical flows, which are usually very close to geostrophic balance --- an instance of balanced flux and source terms --- at all times (see e.g. \citet{gill1982atmosphere}, Section 7.6).

Therefore, a lot of research was conducted over the past two decades to develop so-called \emph{well-balanced} methods which are able to preserve these equilibria exactly. To the best of the author's knowledge, \citet{greenberg1996well} were the first to use the term ``well-balanced''.

Subsequently, dozens of well-balanced methods have been developed, including \citet{leveque1998balancing}, \citet{garcia2000numerical}, \citet{hubbard2000flux}, \citet{burguete2001efficient}, \citet{gascon2001construction}, \citet{rogers2001adaptive}, \citet{bale2003wave}, \citet{rogers2003mathematical}, \citet{audusse2004fast}, \citet{chinnayya2004well}, \citet{liang2009adaptive}, \citet{liang2009numerical}. The most recent articles the author could find are by \citet{zhang2014well} and \citet{chertockwell}, the latter being particularly interesting here, as their assumptions align with those made in this report. Furthermore, Section 4.4 of \citet{zeitlin2007nonlinear} presents a long list of other well-balanced methods applicable to the rotating SWEs and refers to the method discussed in \citet{audusse2004fast} and related works as ``the most classical [well-balanced] method''. With \citet{bouchut2004nonlinear}, there is also a textbook focusing primarily on these methods.

Many of these methods deal with very specific models which are beyond the scope of this report. In particular, some address the use of geometric source terms to model the equations on an irregular grid. Especially more recent papers have largely focused on methods which are capable modelling dry states. Hence, two methods were chosen to be investigated in detail in this report. The method presented in \citet{leveque1998balancing}, which balances the terms by introducing additional Riemann problems, as well as the method due to \citet{rogers2003mathematical}, which employs a change of variables.

Other methods were considered for closer investigation, in particular \citet{hubbard2000flux} and \citet{chertockwell}. However, these essentially develop unsplit balanced methods, which require a considerably different computational framework. Hence, the scope of this report is limited to the above two methods, both of which can be implemented with relatively little effort by adapting existing Godunov-type methods in wave-propagation form.

\section{Report Outline}

The remainder of this report is structured as follows. Chapter~\ref{ch:theory} constitutes the main part of this report and introduces the theory behind these numerical methods and extends them to include the Coriolis terms. Subsequently, chapter~\ref{ch:implementation} describes the test framework which was set up to evaluate these methods as well as implementation details of the methods themselves. Chapter~\ref{ch:results} shows the results obtained from these implementations. Lastly, chapter~\ref{ch:conclusion} draws some conclusions and suggests ways in which further research could improve on the work presented here.